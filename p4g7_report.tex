\documentclass[titlepage]{article}

\usepackage[margin=1.5in]{geometry}
\usepackage{graphicx}
\usepackage{hyperref}
\usepackage{float}
\usepackage{placeins}
\usepackage{amsmath}
\usepackage[noabbrev,nameinlink]{cleveref}

\title{COMS 4444: Project 4, Group 7}
\author{Alpha Kaba, Martin Ristovski, Micah Cheng}
\date{Fall 2022}

\begin{document}

\maketitle

\tableofcontents

\pagebreak

\section{Introduction}

In Project 4: Amoeba, the goal is to design and implement a movement strategy for an amoeba-like agent within a grid whose non-amoeba cells are either empty or contain bacteria.

\subsection{Problem Specification}
We control an amoeba-like creature that occupies a connected set of cells in a 100x100 grid. The grid wraps around horizontally and vertically. Each cell not occupied by the amoeba is either empty or it contains a bacterium, with the bacteria density given by the parameter $d$.

Bacteria are sensitive to their immediate surroundings, and if possible they will try to move away from neighboring bacteria and neighboring amoeba cells. The bacteria move at a speed of one cell per turn. The bacteria will not move if they are surrounded by at least 3 other bacteria or amoeba cells. The bacteria will not move if they are surrounded by empty cells, but they will move if they are surrounded by one or two non-empty cell.

The amoeba also moves at a speed of one cell per turn. Its movement is governed by a parameter called the metabolism, $m$, where $0 < m \leq 1$. On any turn, the amoeba may retract up to $m$ of its cells, and it may extend up to $m$ of its cells. The amoeba may not retract or extend more than one cell in any direction. The amoeba may not retract or extend a cell if it would cause the amoeba to become disconnected.




\section{Initial Implementation}

The first thing that came to mind is that it would be benefitial to have a lot of information about the state of the grid and to have a lot of flexibility in terms of what parts of the amoeba we can move and by how much. This led us to space-filling curves, and from there we decided to implement a modified pseudo-Hilbert curve as the goal shape for our amoeba. This curve is defined as follows:

\begin{itemize}
    \item TODO: define curve
\end{itemize}


\section{Improvements}


\section{Tournament Performance}


\section{Conclusion}

\subsection{Future Work}

\subsection{Acknowledgements}

\subsection{Summary of Contributions}


\end{document}